%===============================================================================
% LaTeX sjabloon voor de bachelorproef toegepaste informatica aan HOGENT
% Meer info op https://github.com/HoGentTIN/bachproef-latex-sjabloon
%===============================================================================

\documentclass{bachproef-tin}

\usepackage{hogent-thesis-titlepage} % Titelpagina conform aan HOGENT huisstijl
\usepackage[export]{adjustbox}
\usepackage{subcaption}
\usepackage{hyperref}

%%---------- Documenteigenschappen ---------------------------------------------
% TODO: Vul dit aan met je eigen info:

% De titel van het rapport/bachelorproef
\title{Europese General Data Protection Regulation voor Software as a service met persoonlijke gegevensverwerking: een praktische analyse}

% Je eigen naam
\author{Frederic Terryn}

% De naam van je promotor (lector van de opleiding)
\promotor{Lieven Smits}

% De naam van je co-promotor. Als je promotor ook je opdrachtgever is en je
% dus ook inhoudelijk begeleidt (en enkel dan!), mag je dit leeg laten.
\copromotor{Gunter Van De Velde}

% Indien je bachelorproef in opdracht van/in samenwerking met een bedrijf of
% externe organisatie geschreven is, geef je hier de naam. Zoniet laat je dit
% zoals het is.
\instelling{Insites Consulting}

% Academiejaar
\academiejaar{2018-2019}

% Examenperiode
%  - 1e semester = 1e examenperiode => 1
%  - 2e semester = 2e examenperiode => 2
%  - tweede zit  = 3e examenperiode => 3
\examenperiode{2}

%===============================================================================
% Inhoud document
%===============================================================================

\begin{document}

%---------- Taalselectie -------------------------------------------------------
% Als je je bachelorproef in het Engels schrijft, haal dan onderstaande regel
% uit commentaar. Let op: de tekst op de voorkaft blijft in het Nederlands, en
% dat is ook de bedoeling!

%\selectlanguage{english}

%---------- Titelblad ----------------------------------------------------------
\inserttitlepage

%---------- Samenvatting, voorwoord --------------------------------------------
\usechapterimagefalse
%%=============================================================================
%% Voorwoord
%%=============================================================================

\chapter*{\IfLanguageName{dutch}{Woord vooraf}{Preface}}
\label{ch:voorwoord}

%% TODO:
%% Het voorwoord is het enige deel van de bachelorproef waar je vanuit je
%% eigen standpunt (``ik-vorm'') mag schrijven. Je kan hier bv. motiveren
%% waarom jij het onderwerp wil bespreken.
%% Vergeet ook niet te bedanken wie je geholpen/gesteund/... heeft


%%=============================================================================
%% Samenvatting
%%=============================================================================

% TODO: De "abstract" of samenvatting is een kernachtige (~ 1 blz. voor een
% thesis) synthese van het document.
%
% Deze aspecten moeten zeker aan bod komen:
% - Context: waarom is dit werk belangrijk?
% - Nood: waarom moest dit onderzocht worden?
% - Taak: wat heb je precies gedaan?
% - Object: wat staat in dit document geschreven?
% - Resultaat: wat was het resultaat?
% - Conclusie: wat is/zijn de belangrijkste conclusie(s)?
% - Perspectief: blijven er nog vragen open die in de toekomst nog kunnen
%    onderzocht worden? Wat is een mogelijk vervolg voor jouw onderzoek?
%
% LET OP! Een samenvatting is GEEN voorwoord!

%%---------- Nederlandse samenvatting -----------------------------------------
%
% TODO: Als je je bachelorproef in het Engels schrijft, moet je eerst een
% Nederlandse samenvatting invoegen. Haal daarvoor onderstaande code uit
% commentaar.
% Wie zijn bachelorproef in het Nederlands schrijft, kan dit negeren, de inhoud
% wordt niet in het document ingevoegd.

\IfLanguageName{english}{%
\selectlanguage{dutch}
\chapter*{Samenvatting}
[1-4]
\selectlanguage{english}
}{}

%%---------- Samenvatting -----------------------------------------------------
% De samenvatting in de hoofdtaal van het document

\chapter*{\IfLanguageName{dutch}{Samenvatting}{Abstract}}

Dit onderzoek is geschreven binnen de context van een Europese regelgeving die sinds mei 2018 van kracht is. Deze regelgeving, de General Data Protection Regulisation (=GDPR), bepaalt de richtlijnen hoe ondernemingen met persoonlijke gegevens moeten omgaan. 

Hoewel de regels eenduidig zijn, is de implementatie hiervan voor elke onderneming anders, afhankelijk van hoe gegevens verwerkt en bijgehouden worden. 

Elke onderneming die gegevens verwerkt, moet voor zichzelf bepalen welke maatregelen er kunnen en moeten getroffen worden om gegevens van gebruikers te beschermen. Dit onderzoek kan daarbij een hulp zijn. Eerst worden algemeen maatregelen voorgesteld die een onderming kan treffen. Daarna wordt een specifiek bedrijf uitgelicht aan de hand van deze maatregelen.

InSites Consulting is een marktonderzoeksbureau dat dagelijks in contact komt met persoonlijke gegevens. Er zijn reeds maatregelen getroffen om te voldoen aan de GDPR, maar er zijn ook nog punten waar verbetering mogelijk is. In dit onderzoek wordt onderzocht hoe die kritieke punten kunnen opgelost worden, en welke methoden hiervoor voorhanden zijn. Dit kan als voorbeeld dienen voor andere ondernemingen die zelf op grote basis gegevens verwerken, en inzichten willen krijgen hoe dit op een veilige manier kan. 

Eén van de kritieke punten die tijdens dit onderzoek naar boven zijn gekomen, is het filteren van persoonlijke gegevens uit niet-structurele data. Dit is een belangrijk item voor InSites Consulting, en is bijgevolgd diepgaand uitgelicht binnen dit onderzoek. Opnieuw kan dit algemeen als voorbeeld dienen voor ondernemingen die ook met dit probleem in contact komen.

Hiervoor zijn aan de hand van bestaande tools, applicaties ontwikkeld die actief persoonlijke data filteren uit niet-gestructureerde data. De ontwikkelede applicaties kunnen hulp bieden bij het automatiseren van dit proces, al moet er echter geconcludeerd worden dit probleem altijd uitzonderingen met zich kan meebrengen, en de tools opgesteld binnen dit onderzoek niet 100 procent sluitend zijn. 

Daarnaast moet worden opgemerkt dat beveiliging van gegevens een continue proces is. De maatregelen voorgesteld in dit onderzoek moeten een stevige basis vormen, maar in de toekomst zullen steeds vragen blijven bestaan naar hoe op de best mogelijke manier kan voldaan worden aan de GDPR. 


%---------- Inhoudstafel -------------------------------------------------------
\pagestyle{empty} % Geen hoofding
\tableofcontents  % Voeg de inhoudstafel toe
\cleardoublepage  % Zorg dat volgende hoofstuk op een oneven pagina begint
\pagestyle{fancy} % Zet hoofding opnieuw aan

%---------- Lijst figuren, afkortingen, ... ------------------------------------

% Indien gewenst kan je hier een lijst van figuren/tabellen opgeven. Geef in
% dat geval je figuren/tabellen altijd een korte beschrijving:
%
%  \caption[korte beschrijving]{uitgebreide beschrijving}
%
% De korte beschrijving wordt gebruikt voor deze lijst, de uitgebreide staat bij
% de figuur of tabel zelf.

\listoffigures
\listoftables

% Als je een lijst van afkortingen of termen wil toevoegen, dan hoort die
% hier thuis. Gebruik bijvoorbeeld de ``glossaries'' package.
% https://www.overleaf.com/learn/latex/Glossaries

%---------- Kern ---------------------------------------------------------------

% De eerste hoofdstukken van een bachelorproef zijn meestal een inleiding op
% het onderwerp, literatuurstudie en verantwoording methodologie.
% Aarzel niet om een meer beschrijvende titel aan deze hoofstukken te geven of
% om bijvoorbeeld de inleiding en/of stand van zaken over meerdere hoofdstukken
% te verspreiden!

%%=============================================================================
%% Inleiding
%%=============================================================================

\chapter{\IfLanguageName{dutch}{Inleiding}{Introduction}}
\label{ch:inleiding}

De inleiding moet de lezer net genoeg informatie verschaffen om het onderwerp te begrijpen en in te zien waarom de onderzoeksvraag de moeite waard is om te onderzoeken. In de inleiding ga je literatuurverwijzingen beperken, zodat de tekst vlot leesbaar blijft. Je kan de inleiding verder onderverdelen in secties als dit de tekst verduidelijkt. Zaken die aan bod kunnen komen in de inleiding~\autocite{Pollefliet2011}:

\begin{itemize}
  \item context, achtergrond
  \item afbakenen van het onderwerp
  \item verantwoording van het onderwerp, methodologie
  \item probleemstelling
  \item onderzoeksdoelstelling
  \item onderzoeksvraag
  \item \ldots
\end{itemize}

\section{\IfLanguageName{dutch}{Probleemstelling}{Problem Statement}}
\label{sec:probleemstelling}

Ondernemingen die wettelijk verplicht zijn te voldoen aan de GDPR moeten sinds mei 2018 op zoek naar alle mogelijke manieren persoonsgegevens die ze bijhouden en verwerken te beveiligen. Hiervoor zijn (zoals verder in dit onderzoek zal blijken) talloze stappen te ondernemen. Voor een onderneming is het belangrijk diepgaand te onderzoeken op welke manieren er kan worden omgesprongen met persoonlijke gegevens, en hoe ze die op een goede manier kunnen/mogen bewaren. 

Algemeen wordt in dit werk gekeken naar KMO's die op geautomatiseerde wijze, op grote\footnote{Wat precies bedoeld wordt met 'groot' leest u verder in dit onderzoek.} schaal gegevens verwerken. 
Voor deze doelgroep kan dit werk een vertrekpunt bieden om een overzicht te krijgen van de mogelijke maatregelen die ze kunnen uitvoeren. 
Specifiek worden deze maatregelen toegepast op InSites Consulting. Er wordt gekeken welke maatregelen reeds uitgevoerd zijn, en wat er nog beter kan. 
Als extra-uitgelicht element wordt bekeken hoe InSites gegevens uit niet-gestructureerde data kan beschermen. 

\section{\IfLanguageName{dutch}{Onderzoeksvraag}{Research question}}
\label{sec:onderzoeksvraag}

Welke maatregelen kunnen op organisatorisch en vooral op technisch (softwarematig) vlak genomen worden om een ondermening te laten voldoen aan de wettelijke normen binnen de GDPR? Aan welke van deze maatregelen voldoet InSites Consulting reeds als onderneming, en wat kan nog beter? Welke oplossingen zijn voorhanden voor de maatregelen die nog niet genomen zijn. 

\section{\IfLanguageName{dutch}{Onderzoeksdoelstelling}{Research objective}}
\label{sec:onderzoeksdoelstelling}

Wat is het beoogde resultaat van je bachelorproef? Wat zijn de criteria voor succes? Beschrijf die zo concreet mogelijk. Gaat het bv. om een proof-of-concept, een prototype, een verslag met aanbevelingen, een vergelijkende studie, enz.

\section{\IfLanguageName{dutch}{Opzet van deze bachelorproef}{Structure of this bachelor thesis}}
\label{sec:opzet-bachelorproef}

% Het is gebruikelijk aan het einde van de inleiding een overzicht te
% geven van de opbouw van de rest van de tekst. Deze sectie bevat al een aanzet
% die je kan aanvullen/aanpassen in functie van je eigen tekst.

De rest van deze bachelorproef is als volgt opgebouwd:

In Hoofdstuk~\ref{ch:stand-van-zaken} wordt een overzicht gegeven van de stand van zaken binnen het onderzoeksdomein, op basis van een literatuurstudie. Hier wordt vooral het theoretisch aspect van de GDPR toegelicht. 

In Hoofdstuk~\ref{ch:Maatregelen} worden de mogelijke maatregelen die een bedrijf kan ondernemen om beter te voldoen aan de GDPR omgeschreven. 

In Hoofdstuk~\ref{ch:methodologie} wordt de methodologie toegelicht en worden de gebruikte onderzoekstechnieken besproken om een antwoord te kunnen formuleren op de onderzoeksvragen.

% TODO: Vul hier aan voor je eigen hoofstukken, één of twee zinnen per hoofdstuk

In Hoofdstuk~\ref{ch:conclusie}, tenslotte, wordt de conclusie gegeven en een antwoord geformuleerd op de onderzoeksvragen. Daarbij wordt ook een aanzet gegeven voor toekomstig onderzoek binnen dit domein.
\chapter{\IfLanguageName{dutch}{Stand van zaken}{State of the art}}
\label{ch:stand-van-zaken}

% Tip: Begin elk hoofdstuk met een paragraaf inleiding die beschrijft hoe
% dit hoofdstuk past binnen het geheel van de bachelorproef. Geef in het
% bijzonder aan wat de link is met het vorige en volgende hoofdstuk.

% Pas na deze inleidende paragraaf komt de eerste sectiehoofding.

Dit hoofdstuk gaat dieper in op de relevante inhoud van de GDPR voor dit onderzoek, en de mogelijke praktische oplossingen. 
Het einddoel is niet om een samenvatting te geven over de theoretische kant van de regelgeving, maar praktische oplossingen te bieden hoe bedrijven hun software 'GDPR-compliant' kunnen maken. 
Het is echter noodzakelijk enkele facetten toe te lichten zodat de lezer de nodige theoretische kennis heeft.
 Het dient als theoretische achtergrond om later (hoofdstuk 3) het effectieve onderzoek uit te voeren.
\section{{Belangrijkste elementen GDPR}}

\subsection{Juridisch}
De Europese commissie beschrijft de juridische aspecten van gegevensbeschermig in \autocite{Eucom2018} als volgt.

\begin{itemize}
    \item \textbf{Grondrecht} Het Handvest van de grondrechten van de EU bepaalt dat EU-burgers recht hebben op bescherming van hun persoonsgegevens.
    \item \textbf{Wetgeving}: 
    \subitem \textbf{De algemene verordening gegevensbescherming} (GDPR): 
    Verordening (EU) 2016/679 betreffende de bescherming van natuurlijke personen in verband met de verwerking van persoonsgegevens en betreffende het vrije verkeer van die gegevens.
    \subitem \textbf{De politierichtlijn}: 
    Richtlijn (EU) 2016/680 betreffende de bescherming van natuurlijke personen in verband met de verwerking van persoonsgegevens door bevoegde autoriteiten met het oog op de voorkoming, het onderzoek, de opsporing en de vervolging van strafbare feiten of de tenuitvoerlegging van straffen, en betreffende het vrije verkeer van die gegevens.
   
\end{itemize}

In dit onderzoek wordt verder gebruik gemaakt van de richtlijnen uitgeschreven binnen de GDPR. 

Verder hebben de EU-landen nationale autoriteiten voor gegevensbescherming aangewezen die dienen toezicht te houden op de bescherming van persoonsgegevens. Voor België is dit: 

\textit{Commission de la protection de la vie privée \\  Rue de la Presse 35 1000 Bruxelles \\  Website: http://www.privacycommission.be/ \\ 
    Art 29 WP Vice-President: Willem DEBEUCKELAERE, President of the Belgian Privacycommission} 

Tot slot is er nog een Europees Comité voor gegevensbescherming. In \textcite{Eucom2018} wordt gesteld dat het comité ruime bevoegdheden heeft om geschillen tussen de nationale toezichthoudende autoriteiten te beslechten, adviezen te geven en richtsnoeren vast te stellen over essentiële aspecten van de algemene verordening gegevensbescherming en de politierichtlijn.

\section{{Opties tot technische implementaties}}

Er zijn heel veel stappen die een bedrijf kan ondernemen om op een veiligere manier om te gaan met hun gegevens. In dit werk wordt gefocust op kmo's die een grote hoeveelheid persoonlijke gegevens verwerken, en hoe die specifiek de software die ze daarvoor gebruiken kunnen aanpassen. 

\subsection{Pseudo- en anonimisatie van gegevens}



\section{{Uitleg Hogent}}
Dit hoofdstuk bevat je literatuurstudie. De inhoud gaat verder op de inleiding, maar zal het onderwerp van de bachelorproef *diepgaand* uitspitten. De bedoeling is dat de lezer na lezing van dit hoofdstuk helemaal op de hoogte is van de huidige stand van zaken (state-of-the-art) in het onderzoeksdomein. Iemand die niet vertrouwd is met het onderwerp, weet nu voldoende om de rest van het verhaal te kunnen volgen, zonder dat die er nog andere informatie moet over opzoeken \autocite{Pollefliet2011}.

Je verwijst bij elke bewering die je doet, vakterm die je introduceert, enz. naar je bronnen. In \LaTeX{} kan dat met het commando \texttt{$\backslash${textcite\{\}}} of \texttt{$\backslash${autocite\{\}}}. Als argument van het commando geef je de ``sleutel'' van een ``record'' in een bibliografische databank in het Bib\LaTeX{}-formaat (een tekstbestand). Als je expliciet naar de auteur verwijst in de zin, gebruik je \texttt{$\backslash${}textcite\{\}}.
Soms wil je de auteur niet expliciet vernoemen, dan gebruik je \texttt{$\backslash${}autocite\{\}}. In de volgende paragraaf een voorbeeld van elk.
\textcite{Lusignan2014} \textcite{ronnie}

\textcite{Knuth1998} schreef een van de standaardwerken over sorteer- en zoekalgoritmen. Experten zijn het erover eens dat cloud computing een interessante opportuniteit vormen, zowel voor gebruikers als voor dienstverleners op vlak van informatietechnologie~\autocite{Creeger2009}.

[7-20]

%%=============================================================================
%% Methodologie
%%=============================================================================

\chapter{\IfLanguageName{dutch}{Methodologie}{Methodology}}
\label{ch:methodologie}

In dit hoofdstuk wordt uitgelegd hoe het toegepaste onderzoek wordt uitgevoerd. Met name, controleren in welke mate een specifieke organisatie (=Insites Consulting) reeds GDPR-compliant is. Er wordt bekeken welke stappen, software-gewijs, nog ondernomen moeten worden. 



\section{Overlopen bulletpoints}
De eerste stap om GDPR compliant te worden is niet de meest igewikkelde, maar des te belangrijk. Stel aan de hand van de uitleg in 'stand van zaken' een lijst op met alle acties die een organisatie kan ondernemen om te voldoen aan de regelgeving. Leg die lijst naast de werking van je organisatie, en beslis wat er kan verbeterd worden, hoe de zaken die nog niet compliant zijn, in volgorde van belang, kunnen opgelost worden.  

\subsection{Data Protection}

%% TODO: Hoe ben je te werk gegaan? Verdeel je onderzoek in grote fasen, en
%% licht in elke fase toe welke stappen je gevolgd hebt. Verantwoord waarom je
%% op deze manier te werk gegaan bent. Je moet kunnen aantonen dat je de best
%% mogelijke manier toegepast hebt om een antwoord te vinden op de
%% onderzoeksvraag.

\subsection{}
Uit dit onderzoek bij InSites Consulting blijkt dus dat ze als organisatie de vraag van een gebruiker om al zijn gegevens niet tot in de puntjes kunnen uitvoeren. Zijn hier mogelijkheden om dat te verbeteren? Dit zien we in punt 2. 

\section{Cognitive Services}
In dit onderdeel van het onderzoek wordt bekeken of het mogelijk is om persoonlijke data te gaan filteren uit open velden waar gebruikers mogelijks persoonlijke informatie achtergelaten hebben. 

\subsection{Azure Api }

[21-25]



% Voeg hier je eigen hoofdstukken toe die de ``corpus'' van je bachelorproef
% vormen. De structuur en titels hangen af van je eigen onderzoek. Je kan bv.
% elke fase in je onderzoek in een apart hoofdstuk bespreken.

%\input{...}
%\input{...}
%...

%%=============================================================================
%% Conclusie
%%=============================================================================

\chapter{Conclusie}
\label{ch:conclusie}

% TODO: Trek een duidelijke conclusie, in de vorm van een antwoord op de
% onderzoeksvra(a)g(en). Wat was jouw bijdrage aan het onderzoeksdomein en
% hoe biedt dit meerwaarde aan het vakgebied/doelgroep? 
% Reflecteer kritisch over het resultaat. In Engelse teksten wordt deze sectie
% ``Discussion'' genoemd. Had je deze uitkomst verwacht? Zijn er zaken die nog
% niet duidelijk zijn?
% Heeft het onderzoek geleid tot nieuwe vragen die uitnodigen tot verder 
%onderzoek?

[76-80]



%%=============================================================================
%% Bijlagen
%%=============================================================================

\appendix
\renewcommand{\chaptername}{Appendix}

%%---------- Onderzoeksvoorstel -----------------------------------------------

\chapter{Onderzoeksvoorstel}

Het onderwerp van deze bachelorproef is gebaseerd op een onderzoeksvoorstel dat vooraf werd beoordeeld door de promotor. Dat voorstel is opgenomen in deze bijlage.

% Verwijzing naar het bestand met de inhoud van het onderzoeksvoorstel
%---------- Inleiding ---------------------------------------------------------

\section{Introductie} % The \section*{} command stops section numbering
\label{sec:introductie}

Hier introduceer je werk. Je hoeft hier nog niet te technisch te gaan.

Je beschrijft zeker:

\begin{itemize}
  \item de probleemstelling en context
  \item de motivatie en relevantie voor het onderzoek
  \item de doelstelling en onderzoeksvraag/-vragen
\end{itemize}

%---------- Stand van zaken ---------------------------------------------------

\section{State-of-the-art}
\label{sec:state-of-the-art}

Hier beschrijf je de \emph{state-of-the-art} rondom je gekozen onderzoeksdomein. Dit kan bijvoorbeeld een literatuurstudie zijn. Je mag de titel van deze sectie ook aanpassen (literatuurstudie, stand van zaken, enz.). Zijn er al gelijkaardige onderzoeken gevoerd? Wat concluderen ze? Wat is het verschil met jouw onderzoek? Wat is de relevantie met jouw onderzoek?

Verwijs bij elke introductie van een term of bewering over het domein naar de vakliteratuur, bijvoorbeeld~\autocite{Doll1954}! Denk zeker goed na welke werken je refereert en waarom.

% Voor literatuurverwijzingen zijn er twee belangrijke commando's:
% \autocite{KEY} => (Auteur, jaartal) Gebruik dit als de naam van de auteur
%   geen onderdeel is van de zin.
% \textcite{KEY} => Auteur (jaartal)  Gebruik dit als de auteursnaam wel een
%   functie heeft in de zin (bv. ``Uit onderzoek door Doll & Hill (1954) bleek
%   ...'')

Je mag gerust gebruik maken van subsecties in dit onderdeel.

%---------- Methodologie ------------------------------------------------------
\section{Methodologie}
\label{sec:methodologie}

Hier beschrijf je hoe je van plan bent het onderzoek te voeren. Welke onderzoekstechniek ga je toepassen om elk van je onderzoeksvragen te beantwoorden? Gebruik je hiervoor experimenten, vragenlijsten, simulaties? Je beschrijft ook al welke tools je denkt hiervoor te gebruiken of te ontwikkelen.

%---------- Verwachte resultaten ----------------------------------------------
\section{Verwachte resultaten}
\label{sec:verwachte_resultaten}

Hier beschrijf je welke resultaten je verwacht. Als je metingen en simulaties uitvoert, kan je hier al mock-ups maken van de grafieken samen met de verwachte conclusies. Benoem zeker al je assen en de stukken van de grafiek die je gaat gebruiken. Dit zorgt ervoor dat je concreet weet hoe je je data gaat moeten structureren.

%---------- Verwachte conclusies ----------------------------------------------
\section{Verwachte conclusies}
\label{sec:verwachte_conclusies}

Hier beschrijf je wat je verwacht uit je onderzoek, met de motivatie waarom. Het is \textbf{niet} erg indien uit je onderzoek andere resultaten en conclusies vloeien dan dat je hier beschrijft: het is dan juist interessant om te onderzoeken waarom jouw hypothesen niet overeenkomen met de resultaten.



%%---------- Andere bijlagen --------------------------------------------------
% TODO: Voeg hier eventuele andere bijlagen toe
%\input{...}

%%---------- Referentielijst --------------------------------------------------

\printbibliography[heading=bibintoc]

\end{document}
