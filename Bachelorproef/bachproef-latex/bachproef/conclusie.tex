%%=============================================================================
%% Conclusie
%%=============================================================================

\chapter{Conclusie}
\label{ch:conclusie}

% TODO: Trek een duidelijke conclusie, in de vorm van een antwoord op de
% onderzoeksvra(a)g(en). Wat was jouw bijdrage aan het onderzoeksdomein en
% hoe biedt dit meerwaarde aan het vakgebied/doelgroep? 
% Reflecteer kritisch over het resultaat. In Engelse teksten wordt deze sectie
% ``Discussion'' genoemd. Had je deze uitkomst verwacht? Zijn er zaken die nog
% niet duidelijk zijn?
% Heeft het onderzoek geleid tot nieuwe vragen die uitnodigen tot verder 
%onderzoek?
Er is onderzocht welke organisatorische en technische maatregelen een onderneming kan treffen om te voldoen aan de regels opgesteld binnen de GDPR. 

Dit onderzoek wijst uit dat er brede en algemene maatregelen mogelijk zijn bij het beveiligen van persoonlijke data, zoals data security en data retentie. 
Er is gewezen op het belang van beveiliging van de software waarmee er wordt gewerkt, in combinatie met de sensibilisering van het personeel. 

Naast algemene beveiliging is ook gebleken dat de GDPR specifieke eisen stelt. Zoals het aanstellen van een persoon die verantwoordelijk is voor de beveiliging. En zoals het taalgebruik gehanteerd in bij het opvragen van gegevens, en bij het geven van toestemming als gebruiker om persoonlijke gegevens te verwerken. 
Er zijn maatregelen voorgesteld om specifiekere problemen, zoals het beveiligen van een Excel-bestand, of data niet herleidbaar te maken tot personen, op te lossen  

Tijdens het onderzoek zijn zodus veel maatregelen gevonden voor ondernemingen om aan de GDPR te voldoen. Maar er is ook gebleken dat deze maatregelen specifieke implementatie vragen van bedrijf tot bedrijf, en elke onderneming zelf moet afwegen of er nog aanvullende maatregelen nodig zijn. 

Deze afweging is gemaakt voor het bedrijf Insites Consulting. Er is een vergelijking gemaakt van de voorgestelde maatregelen in dit onderzoek, en de maatregelen reeds getroffen binnen het bedrijf. Hierbij zijn enkele specifieke tekortkomingen naar boven gekomen, zoals het vinden van persoonlijke gegevens in niet-gestructureerde data. 

Dit probleem en de mogelijke oplossingen hiervoor zijn dieper uitgelicht. Als resultaat zijn enkele applicaties ontwikkeld die InSites Consulting in de toekomst kunnen helpen. Er moet echter opgemerkt worden dat dit probleem uitzonderingen met zich meebrengt, en bijgevolg de ontwikkelde applicaties niet volledig sluitend zijn. Daarnaast is het probleem naar boven gekomen dat de gebruikte tools van Microsoft enkel Engelstalige ondersteuning bieden, wat niet voldoende is voor een internationaal bedrijf als InSites Consulting. 

 



