%%=============================================================================
%% Samenvatting
%%=============================================================================

% TODO: De "abstract" of samenvatting is een kernachtige (~ 1 blz. voor een
% thesis) synthese van het document.
%
% Deze aspecten moeten zeker aan bod komen:
% - Context: waarom is dit werk belangrijk?
% - Nood: waarom moest dit onderzocht worden?
% - Taak: wat heb je precies gedaan?
% - Object: wat staat in dit document geschreven?
% - Resultaat: wat was het resultaat?
% - Conclusie: wat is/zijn de belangrijkste conclusie(s)?
% - Perspectief: blijven er nog vragen open die in de toekomst nog kunnen
%    onderzocht worden? Wat is een mogelijk vervolg voor jouw onderzoek?
%
% LET OP! Een samenvatting is GEEN voorwoord!

%%---------- Nederlandse samenvatting -----------------------------------------
%
% TODO: Als je je bachelorproef in het Engels schrijft, moet je eerst een
% Nederlandse samenvatting invoegen. Haal daarvoor onderstaande code uit
% commentaar.
% Wie zijn bachelorproef in het Nederlands schrijft, kan dit negeren, de inhoud
% wordt niet in het document ingevoegd.

\IfLanguageName{english}{%
\selectlanguage{dutch}
\chapter*{Samenvatting}
[1-4]
\selectlanguage{english}
}{}

%%---------- Samenvatting -----------------------------------------------------
% De samenvatting in de hoofdtaal van het document

\chapter*{\IfLanguageName{dutch}{Samenvatting}{Abstract}}

Dit onderzoek is geschreven binnen de context van een Europese regelgeving die sinds mei 2018 van kracht is. Deze regelgeving, de General Data Protection Regulisation (=GDPR), bepaalt de richtlijnen hoe ondernemingen met persoonlijke gegevens moeten omgaan. 

Hoewel de regels eenduidig zijn, is de implementatie hiervan voor elke onderneming anders, afhankelijk van hoe gegevens verwerkt en bijgehouden worden. 

Elke onderneming die gegevens verwerkt, moet voor zichzelf bepalen welke maatregelen er kunnen en moeten getroffen worden om gegevens van gebruikers te beschermen. Dit onderzoek kan daarbij een hulp zijn. Eerst worden algemeen maatregelen voorgesteld die een onderming kan uitvoeren. Daarna wordt een specifiek bedrijf uitgelicht aan de hand van deze maatregelen.

InSites Consulting is een marktonderzoeksbureau dat dagelijks in contact komt met persoonlijke gegevens. Er zijn reeds maatregelen getroffen om te voldoen aan de GDPR, maar er zijn ook nog punten waar verbetering mogelijk is. In dit onderzoek wordt onderzocht hoe die kritieke punten kunnen opgelost worden, en welke methoden hiervoor voorhanden zijn. Dit kan als voorbeeld dienen voor andere ondernemingen die zelf op grote basis gegevens verwerken, en inzichten willen krijgen hoe dit op een veilige manier kan. 

Eén van de kritieke punten die tijdens dit onderzoek naar boven zijn gekomen, is het filteren van persoonlijke gegevens uit niet-structurele data. Dit is een belangrijk item voor InSites Consulting, en is bijgevolgd diepgaand uitgelicht binnen dit onderzoek. Opnieuw kan dit algemeen als voorbeeld dienen voor ondernemingen die ook met dit probleem in contact komen.

Hiervoor zijn aan de hand van bestaande tools, applicaties ontwikkeld die geautomatiseerd persoonlijke data filteren uit niet-gestructureerde data. De gebruikte tools hiervoor zijn de cognitive services van Microsoft. Er kan geconcludeerd worden dat de ontwikkelde applicatie op een goede en betrouwbare manier foto's kan filteren die persoonlijke informatie bevatten. 
Het filteren van persoonlijke informatie uit tekst is iets minder succesvol gebleken. De ontwikkelede applicaties kunnen een hulp bieden bij het automatiseren van het proces, maar zijn niet 100 procent betrouwbaar. De beschikbare text-analyse tools van Microsoft voldoen nog niet volledig aan de vereisten om een sluitende applicatie te ontwikkelen. 

Daarnaast moet worden opgemerkt dat beveiliging van gegevens een continue proces is. De maatregelen voorgesteld in dit onderzoek moeten een stevige basis vormen, maar in de toekomst zullen steeds vragen blijven bestaan naar hoe op de best mogelijke manier kan voldaan worden aan de GDPR. 
