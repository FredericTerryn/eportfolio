\chapter{\IfLanguageName{dutch}{Stand van zaken}{State of the art}}
\label{ch:stand-van-zaken}

% Tip: Begin elk hoofdstuk met een paragraaf inleiding die beschrijft hoe
% dit hoofdstuk past binnen het geheel van de bachelorproef. Geef in het
% bijzonder aan wat de link is met het vorige en volgende hoofdstuk.

% Pas na deze inleidende paragraaf komt de eerste sectiehoofding.

Dit hoofdstuk gaat dieper in op de relevante inhoud van de GDPR voor dit onderzoek, en de mogelijke praktische oplossingen. 
Het einddoel is niet om een samenvatting te geven over de theoretische kant van de regelgeving, maar praktische oplossingen te bieden hoe bedrijven hun software 'GDPR-compliant' kunnen maken. 
Het is echter noodzakelijk enkele facetten toe te lichten zodat de lezer de nodige theoretische kennis heeft.
 Het dient als theoretische achtergrond om later (hoofdstuk 3) het effectieve onderzoek uit te voeren.
\section{{Belangrijkste elementen GDPR}}

\subsection{Juridisch}
De Europese commissie beschrijft de juridische aspecten van gegevensbeschermig in \autocite{Eucom2018} als volgt.

\begin{itemize}
    \item \textbf{Grondrecht} Het Handvest van de grondrechten van de EU bepaalt dat EU-burgers recht hebben op bescherming van hun persoonsgegevens.
    \item \textbf{Wetgeving}: 
    \subitem \textbf{De algemene verordening gegevensbescherming} (GDPR): 
    Verordening (EU) 2016/679 betreffende de bescherming van natuurlijke personen in verband met de verwerking van persoonsgegevens en betreffende het vrije verkeer van die gegevens.
    \subitem \textbf{De politierichtlijn}: 
    Richtlijn (EU) 2016/680 betreffende de bescherming van natuurlijke personen in verband met de verwerking van persoonsgegevens door bevoegde autoriteiten met het oog op de voorkoming, het onderzoek, de opsporing en de vervolging van strafbare feiten of de tenuitvoerlegging van straffen, en betreffende het vrije verkeer van die gegevens.
   
\end{itemize}

In dit onderzoek wordt verder gebruik gemaakt van de richtlijnen uitgeschreven binnen de GDPR. 

Verder hebben de EU-landen nationale autoriteiten voor gegevensbescherming aangewezen die dienen toezicht te houden op de bescherming van persoonsgegevens. Voor België is dit: 

\textit{Commission de la protection de la vie privée \\  Rue de la Presse 35 1000 Bruxelles \\  Website: http://www.privacycommission.be/ \\ 
    Art 29 WP Vice-President: Willem DEBEUCKELAERE, President of the Belgian Privacycommission} 

Tot slot is er nog een Europees Comité voor gegevensbescherming. In \textcite{Eucom2018} wordt gesteld dat het comité ruime bevoegdheden heeft om geschillen tussen de nationale toezichthoudende autoriteiten te beslechten, adviezen te geven en richtsnoeren vast te stellen over essentiële aspecten van de algemene verordening gegevensbescherming en de politierichtlijn.

\section{{Opties tot technische implementaties}}

Er zijn heel veel stappen die een bedrijf kan ondernemen om op een veiligere manier om te gaan met hun gegevens. In dit werk wordt gefocust op kmo's die een grote hoeveelheid persoonlijke gegevens verwerken, en hoe die specifiek de software die ze daarvoor gebruiken kunnen aanpassen. 

\subsection{Pseudo- en anonimisatie van gegevens}



\section{{Uitleg Hogent}}
Dit hoofdstuk bevat je literatuurstudie. De inhoud gaat verder op de inleiding, maar zal het onderwerp van de bachelorproef *diepgaand* uitspitten. De bedoeling is dat de lezer na lezing van dit hoofdstuk helemaal op de hoogte is van de huidige stand van zaken (state-of-the-art) in het onderzoeksdomein. Iemand die niet vertrouwd is met het onderwerp, weet nu voldoende om de rest van het verhaal te kunnen volgen, zonder dat die er nog andere informatie moet over opzoeken \autocite{Pollefliet2011}.

Je verwijst bij elke bewering die je doet, vakterm die je introduceert, enz. naar je bronnen. In \LaTeX{} kan dat met het commando \texttt{$\backslash${textcite\{\}}} of \texttt{$\backslash${autocite\{\}}}. Als argument van het commando geef je de ``sleutel'' van een ``record'' in een bibliografische databank in het Bib\LaTeX{}-formaat (een tekstbestand). Als je expliciet naar de auteur verwijst in de zin, gebruik je \texttt{$\backslash${}textcite\{\}}.
Soms wil je de auteur niet expliciet vernoemen, dan gebruik je \texttt{$\backslash${}autocite\{\}}. In de volgende paragraaf een voorbeeld van elk.
\textcite{Lusignan2014} \textcite{ronnie}

\textcite{Knuth1998} schreef een van de standaardwerken over sorteer- en zoekalgoritmen. Experten zijn het erover eens dat cloud computing een interessante opportuniteit vormen, zowel voor gebruikers als voor dienstverleners op vlak van informatietechnologie~\autocite{Creeger2009}.

[7-20]
