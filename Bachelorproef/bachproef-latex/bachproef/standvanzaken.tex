% !TeX spellcheck = nl_NL
\chapter{\IfLanguageName{dutch}{Stand van zaken}{State of the art}}
\label{ch:stand-van-zaken}

% Tip: Begin elk hoofdstuk met een paragraaf inleiding die beschrijft hoe
% dit hoofdstuk past binnen het geheel van de bachelorproef. Geef in het
% bijzonder aan wat de link is met het vorige en volgende hoofdstuk.

% Pas na deze inleidende paragraaf komt de eerste sectiehoofding.

Dit hoofdstuk gaat dieper in op de relevante inhoud van de GDPR voor dit onderzoek, en de mogelijke praktische oplossingen. 
Het einddoel is niet om een samenvatting te geven over de theoretische kant van de regelgeving, maar praktische oplossingen te bieden hoe bedrijven hun software 'GDPR-compliant' kunnen maken. 
Het is echter noodzakelijk enkele facetten toe te lichten zodat de lezer de nodige theoretische kennis heeft.
 Het dient als theoretische achtergrond om in hoofdstuk 3 het effectieve onderzoek uit te voeren.
 
 
\section{{GDPR Theoretisch}}

De EU General Data Protection Regulation (GDPR) is een regelgeving ontworpen door de EU, met als bedoeling een eerste stap te bieden naar het teruggeven van controle aan de EU-burgers, over wat er met hun persoonlijke gevegens gebeurt. 

\subsection{Juridisch}
De Europese commissie beschrijft de juridische aspecten van gegevensbescherming in \autocite{Eucom2018} als volgt.

\begin{itemize}
    \item \textbf{Grondrecht} Het Handvest van de grondrechten van de EU bepaalt dat EU-burgers recht hebben op bescherming van hun persoonsgegevens.
    \item \textbf{Wetgeving}: 
    \subitem \textbf{De algemene verordening gegevensbescherming} (GDPR): 
    Verordening (EU) 2016/679 betreffende de bescherming van natuurlijke personen in verband met de verwerking van persoonsgegevens en betreffende het vrije verkeer van die gegevens.
    \subitem \textbf{De politierichtlijn}: 
    Richtlijn (EU) 2016/680 betreffende de bescherming van natuurlijke personen in verband met de verwerking van persoonsgegevens door bevoegde autoriteiten met het oog op de voorkoming, het onderzoek, de opsporing en de vervolging van strafbare feiten of de tenuitvoerlegging van straffen, en betreffende het vrije verkeer van die gegevens.
   
\end{itemize}

In dit onderzoek wordt verder gebruik gemaakt van de richtlijnen uitgeschreven binnen de GDPR. 

Verder hebben de EU-landen nationale autoriteiten voor gegevensbescherming aangewezen die dienen toezicht te houden op de bescherming van persoonsgegevens. Voor België is dit: 

\quad \textit{Gegevensbeschermingsautoriteit / Commission de la protection de la vie privée. \\ \quad Rue de la Presse 35 1000 Brussel \\  \quad Website: http://www.privacycommission.be/ \\ 
    \quad Art 29 WP Vice-President: Willem DEBEUCKELAERE, President of the Belgian Privacycommission} 

\begin{figure}[h]
	\centering
	\begin{subfigure}{0.4\textwidth}
		\centering
		\includegraphics[width=.4\linewidth]{gba.png}
		\caption{Gegevensbeschermingsautoriteit.}
	\end{subfigure}
	\begin{subfigure}{0.5\textwidth}
		\centering
		\includegraphics[width=.4\linewidth]{CPVP.jpg}
		\caption{Commission de la vie privée}
	\end{subfigure}%
	\caption{Autoriteit voor gegevensbescherming in België. (Vlaanderen en Wallonië)}
\end{figure}

Tot slot is er nog een Europees Comité voor gegevensbescherming. In \textcite{Eucom2018} wordt gesteld dat het comité ruime bevoegdheden heeft om geschillen tussen de nationale toezichthoudende autoriteiten te beslechten, adviezen te geven en richtsnoeren vast te stellen over essentiële aspecten van de algemene verordening gegevensbescherming en de politierichtlijn.

%europa.eu/european-union/about-eu/institutions-bodies/european-data-protection-supervisor_nl)


Hierbinnen is een Europees toezichthouder aangesteld voor de gegevensbescherming. Zijn rol is om erop toe te zien dat de EU-instellingen en -organen de privacy van de burgers respecteren bij de verwerking van persoonsgegevens. Op moment van schrijven is dit \textit{Giovanni Buttarelli}. 
\\ Hij controleert dus niet zozeer zelfstandige organisaties, maar de instellingen die gegevens verwerken. 


\subsection{Voor wie? }
Niet elke organisatie moet wakker liggen van de GDPR. Als je geen gegevens verwerkt, of geen gegevens bijhoudt van gebruikers kom je er helemaal niet mee in contact. 
Er zijn bepaalde voorwaarden geformuleerd. 

\subsubsection{Materieel toepassingsgebied.}
Artikel 2 van de GDPR spreekt over het materieel toepassingsgebied. Aangezien het artikel in rechtstaal geschreven is, brengt een definitie van de GBA (gegevensbeschermingsautoriteit) meer duidelijkheid. 

\textit{Deze verordening is van toepassing op de geheel of gedeeltelijk geautomatiseerde verwerking, alsmede op de niet-geautomatiseerde verwerking van persoonsgegevens die in een bestand zijn opgenomen of die bestemd zijn om daarin te worden opgenomen.}

\subsubsection{Territoriaal toepassingsgebied}
In welk deel van de wereld van toepassing? In de eerse plaats is het een Europese regelgeving. Dit betekent echter niet dat er bij niet-Europese organisaties geen rekening mee moet gehouden worden. Er zijn twee gevallen te onderscheiden: 

\begin{itemize}
	\item De verwerkingsverantwoordelijke is gevestigd in de EU. 
	\subitem Lorem ipsum dolor sit amet, consectetur adipiscing elit, sed do eiusmod tempor incididunt ut labore et dolore magna aliqua. Ut enim ad minim veniam, quis nostrud exercitation ullamco laboris nisi ut aliquip ex ea commodo consequat. 
	\item De verwerkingsverantwoordelijke is gevestigd buiten de EU. 
	\subitem Lorem ipsum dolor sit amet, consectetur adipiscing elit, sed do eiusmod tempor incididunt ut labore et dolore magna aliqua. Ut enim ad minim veniam, quis nostrud exercitation ullamco laboris nisi ut aliquip ex ea commodo consequat. 
\end{itemize}

//NOTE TO SELF DIT MOET UITGEBREIDER, ER ZIJN MEER VOORWAARDEN 


\subsection{{Begrippen: GDPR}}
http://www.privacy-regulation.eu/nl/artikel-4-definities-EU-AVG.htm
Verder in dit onderzoek zullen specifieke termen die binnen de GDPR beschreven staan gebruikt worden.
In artikel 4 van de GDPR vind je de juridische definities van die termen. Hier volgt wat extra toelichting bij de belangrijkste. 

\subsubsection{Persoonsgegevens} 
Definitie: alle informatie over een geïdentificeerde of identificeerbare natuurlijke persoon ("\textbf{de betrokkene}"); als identificeerbaar wordt beschouwd een natuurlijke persoon die direct of indirect kan worden geïdentificeerd, met name aan de hand van een identificator zoals een naam, een identificatienummer, locatiegegevens, een online identificator of van een of meer elementen die kenmerkend zijn voor de fysieke, fysiologische, genetische, psychische, economische, culturele of sociale identiteit van die natuurlijke persoon.

 Als verder in dit onderzoek wordt gesproken over persoonlijke data, persoonsgegevens, persoonlijke gegevens, etc., gaat het over data die bovenstaande definitie volgt.
 
  Merk op:
  \\ Ook gegevens die indirect een persoon kunnen identificeren worden als persoonlijke info beschouwd. 
 
 Voorbeeld: school, hobby, leeftijd
 \\ Deze drie gegevens zijn op zichzelf, alleenstaand, niet genoeg om een persoon te identificeren.
 Maar als we de drie nu combineren, kan het in bepaalde gevallen leiden tot identificatie van een persoon. Stel, hij/zij gaat naar een kleine dorpsschool, is 9 jaar en beoefent judo. In veel gevallen zal dit te herleiden zijn tot één persoon, en worden dit indirect persoonlijke gegevens. 

Er zal vaak ter sprake komen dat de persoonsgegevens \underline{\textit{verwerkt}} worden, de definitie hiervan luidt als volgt: een bewerking of een geheel van bewerkingen met betrekking tot persoonsgegevens of een geheel van persoonsgegevens, al dan niet uitgevoerd via geautomatiseerde procedés, zoals het verzamelen, vastleggen, ordenen, structureren, opslaan, bijwerken of wijzigen, opvragen, raadplegen, gebruiken, verstrekken door middel van doorzending, verspreiden of op andere wijze ter beschikking stellen, aligneren of combineren, afschermen, wissen of vernietigen van gegevens.

In het vervolg van dit onderzoek zal regelmatig verwezen wroden naar persoonlijke gegevens met \textbf{PII}. Dit is de afkorting voor Personally Identifiable Information.

\subsubsection{Pseudonimisering} 
Het verwerken van persoonsgegevens op zodanige wijze dat de persoonsgegevens niet meer aan een specifieke betrokkene kunnen worden gekoppeld zonder dat er aanvullende gegevens worden gebruikt, mits deze aanvullende gegevens apart worden bewaard en technische en organisatorische maatregelen worden genomen om ervoor te zorgen dat de persoonsgegevens niet aan een geïdentificeerde of identificeerbare natuurlijke persoon worden gekoppeld. 

\subsubsection{Anonimiseren}

\subsubsection{Toestemming} 
De regelgeving is er in de eerste plaats gekomen om mensen te beschermen. Al te vaak werd in het verleden op een dubieuze manier toestemming gevraagd voor de verwerking van persoonsgegevens, en verstonden de betrokken gebruikers niet wat dat allemaal inhield. Daarom stelt de EU nu dat gebruikers heel duidelijk moeten weten welke data ze vrijgeven en wat er met die data mag/zal gedaan worden, en dit moet gepresenteerd worden in ondubbelizinnige, begrijpbare taal.
Darnaast moet toestemming specifiek zijn.
\\ Als voorbeeld: een hokje om aan te vinken: “Hierbij stem ik in om al mijn persoonlijke info ter beschikking te stellen aan de organisatie”, is geen enkel geval specifiek genoeg, en verboden.\\ Definitie \textit{toestemming van de betrokkene}: elke vrije, specifieke, geïnformeerde en ondubbelzinnige wilsuiting waarmee de betrokkene door middel van een verklaring of een ondubbelzinnige actieve handeling hem betreffende verwerking van persoonsgegevens aanvaardt. 


\subsubsection{Inbreuken} 
Het moet ten alle koste vermeden worden, maar het is niet uitgesloten dat bepaalde persoonsgegevens ongewild worden verspreidt. Denk aan hackers, data-leaks, enzoverder.
\\ De definitie wordt alsvolgt gesteld: een inbreuk op de beveiliging die per ongeluk of op onrechtmatige wijze leidt tot de vernietiging, het verlies, de wijziging of de ongeoorloofde verstrekking van of de ongeoorloofde toegang tot doorgezonden, opgeslagen of anderszins verwerkte gegevens; 

\subsection{Begrippen: gebruikt binnen dit onderzoek}
API, machine learning, JSON, SDK 

\subsection{Rollen}
Binnen de GDPR zij verschillende partijen te onderscheiden. Neem als voorbeeld een KMO die gegevens van gebruikers verzamelt. Dan valt een onderscheid te maken in rol tussen de persoon van wie de gegevens verwerkt worden, en de verantwoordelijke binnen de organisatie die de gegevens verwerkt. Maar als er binnen de kmo ook gegevens worden bijgehouden van personeelsleden valt dit ook onder de regelgeving. Dat zijn dus al drie totaal verschillende rollen van waaruit je in contact komt met de GDPR. 

De belangrijkste rollen worden hieronder beschreven. 

\subsubsection{Betrokkene}
De definitie van de betrokkene staat beschreven onder 2.1.3 Begrippen; persoonsgegevens. Dit stelt de gebruiker voor van wie een organisatie de gegevens wil bijhouden of verwerken. De GDPR is er in de eerste plaats om de betrokkene meer rechten te geven. //NOTE TO SELF, MEER UITLEG HIER? 

\subsubsection{Verwerkingsverantwoordelijke}
De voornaamste rol van de verwerkingsverantwoordelijke is het bepalen van de doelen en redenen voor het verwerken van de persoonlijke gegevens. Dit kunnen in realiteit meerdere mensen zijn. Het belangrijkste is dat deze persoon verantwoordelijk wordt geacht acties te ondernemen om het verzamelen en verwerken van data GDPR-compliant te maken. 
Dit kan bijvoorbeeld door het aanstellen van een DPO (zie verder), en is in praktijk vaak de zaakvoerder.  

\subsubsection{DPO}
Data Protection Officer. Een verantwoordelijke aangesteld om de strategie te bepalen hoe data op een goede en veilige manier kan verzameld worden. Hij houdt ook toezicht op de implementatie van deze strategie.
Enkel onder bepaalde voorwaarden (zie bijlage) is het aanstellen van een DPO verplicht, maar het is ten zeerste aan te raden zodat de verantwoordelijkheid duidelijk bij een bepaald persoon ligt. 

\subsubsection{Bevoegde autoriteiten}
Zie 2.1.1. //OOK MEER UITLEG HIER 




\subsection{Rechtmatigheid van verwerking} www.privacy-regulation.eu
Als organisatie of rechtspersoon in het algemeen moet je er van uitgaan dat het verzamelen van persoonsgegevens standaard verboden is. Als je er toch gebruik van wilt maken, moet aan minstens één van de volgende voorwaarden voldaan worden. 
\begin{itemize}
    \item \textbf{Toestemming:} Gegevensverwerking kan enkel indien er de betrokkene expliciet toestemming (zie sectie 2.1.2) heeft gegeven om (beperkt) met zijn persoonlijke data aan de slag te gaan en die te verwerken. \\
    
    \item \textbf{Noodzaak}: De verwerking is noodzakelijk voor de uitvoering van een overeenkomst waarbij de betrokkene partij is, of om op verzoek van de betrokkene vóór de sluiting van de overeenkomst maatregelen te nemen. \\
    
    \item \textbf{Voldoen aan wettelijke verplichting }: de verwerking is noodzakelijk om te voldoen aan een wettelijke verplichting die op de verwerkingsverantwoordelijke rust; \\
    
     \item \textbf{Bescherming van vitale belangen}:  de verwerking is noodzakelijk om de vitale belangen van de betrokkene of van een andere natuurlijke persoon te beschermen. \\
    
    \item \textbf{Algemeen belang}: de verwerking is noodzakelijk voor de vervulling van een taak van algemeen belang of van een taak in het kader van de uitoefening van het openbaar gezag dat aan de verwerkingsverantwoordelijke is opgedragen. \\
    
     \item \textbf{Behartiging van de belangen}: de verwerking is noodzakelijk voor de behartiging van de gerechtvaardigde belangen van de verwerkingsverantwoordelijke of van een derde, behalve wanneer de belangen of de grondrechten en de fundamentele vrijheden van de betrokkene die tot bescherming van persoonsgegevens nopen, zwaarder wegen dan die belangen, met name wanneer de betrokkene een kind is. \\
\end{itemize}

\begin{figure}[h]
    \centering
    \includegraphics[scale=0.7]{six_pilars_processing.jpg}
    \caption{6 voorwaarden om rechtmatig gegevens te verwerken, schematische voorstelling.}
    https://www.i-scoop.eu/gdpr/legal-grounds-lawful-processing-personal-data/
\end{figure}


\section{Organisatorische en technische maatregelen.}
\subsection{onderscheid gestructureerde data en ongestructureerde data}




\subsection{Toestemming van de gebruiker - dit deel staat niet echt op zijn plaats}
In sectie 2.1.2 wordt toegelicht wat het volgens de GDPR begrepen wordt onder toestemming van de gebruiker. Het is één van de zes opties om te kunnen overgaan tot gegevensverwerking. De GDPR focust vooral op de begrijpelijkheid en duidelijkheid waar de gebruiker toestemming tot geeft. Er zijn enkele voorwaarden voor toestemming opgenomen in de wetgeving. 
\begin{itemize}
    \item  Als je gegevens verwerkt op basis van toestemming, moet je als organisatie ten alle tijde bewijs kunnen leveren dat de betrokkene specifiek voor elke soort gegevens toestemming heeft gegeven. 
    \item  Het verzoek dient in een \textbf{begrijpelijke en gemakkelijk toegankelijke vorm en in duidelijke en eenvoudige taal} opgesteld te zijn. 
    \item  
    De toestemming kan ten alle tijde \textbf{ingetrokken worden}, en dit proces is even eenvoudig als het het geven ervan. 
\end{itemize}




\subsection{Voldoen aan de rechten van de betrokkene}

Eén van de hoofdonderwerpen die de GDPR aansnijdt zijn de rechten van de betrokkene in verband met zijn gegevens. Zelfs al heeft die persoon oorspronkelijk toestemming gegeven om zijn gegevens te verzamelen en verwerken behoudt de volgende rechten.


\subsection{klad nog te verwerken: }
.6.1 Recht van inzage
Indien een verwerkingsverantwoordelijke de persoonsgegevens van een betrokkene verwerkt, heeft deze laatste volgens artikel 15 van de EU (2016) het recht deze gegevens in te
zien en hieromtrent extra informatie op te vragen.
De verwerkingsverantwoordelijke dient hier kosteloos gevolg aan te geven, tenzij de
verzoeken van de betrokkene ongegrond of buitensporig zijn. Bij repetitieve herhaling
mag de verwerkingsverantwoordelijke hiervoor een administratieve vergoeding vragen of
weigeren gevolg te geven aan het verzoek (EU, 2016, art. 12.5).

\chapter{\IfLanguageName{dutch}{Maatregelen}{}}
\label{ch:Maatregelen}

In vorig hoofdstuk is het theoretisch aspect van de GDPR uitgelegd. In dit hoofdstuk wordt dieper ingegaan op welke maatregelen precies kunnen genomen worden om een onderneming te laten voldoen aan de regelgeving. Er worden \textbf{organisatorische} maatregelen besproken, maar de nadruk ligt op de \textbf{technische maatregelen}.   

In de regelgeving wordt een verschil gemaakt tussen verschillende types organisaties. Bijvoorbeeld: of er al dan niet in grote mate persoonlijke gegevens verwerkt worden. Ahankelijk hiervan zijn de regels minder streng/strenger. In dit werk wordt gefocust op kmo's die een grote hoeveelheid persoonlijke gegevens verwerken. 

\section{Organisatorische en Technische maatregelen}

\subsection{Privacy Policy mededeling}
De meeste organisaties beschikken over een privacy policy. Dit dient (binnen de scope van de GDPR) om de gebruiker, vooraleer hij/zij persoonlijke info achterlaat, te informeren in welke mate die gegevens zullen bijgehouden worden, en hoe ze worden verwerkt. De GDPR stelt echter enkele nieuwe voorwaarden om de gebruiker te beschermen, vooral tegen het feit dat deze policies vaak onduidelijk zijn, en de gebruiker niet weet waarmee hij akkoord gaat.
Artikel 12, 13 en 14 (VERWIJZING) van de GDPR geven uitgebreide instructies hoe privacy policies opgesteld moeten worden, en op die inhoud zal niet tot in detail worden ingegaan tijdens dit onderzoek; een goede privacy policy kan wel worden omschreven in vier eigenschappen. 

\begin{itemize}
	\item Moet geschreven zijn in beknopte, transparante, begrijpelijke en toegankelijke vorm.
	\item Moet duidelijke en ondubbelzinnige taal bevatten, die eenduidig te interepreteren is. 
	\item Moet gratis aangeboden worden.
	\item Moet tijdig aangeboden worden.
\end{itemize}

Er zijn softwarematig verschillende manieren om privacy policy op een goede manier weer te geven die worden besproken in de sectie 4: Methodologie. 

\subsection{Data security}
Naast duidelijkheid en openheid over hoe je als organisatie gegevens verwerkt, moet je er ook voor zorgen dat de resultaten hiervan, en bij uitbreiding alle gegevens die je opslaat, binnen je organisatie blijven.
Maatregelen die hier moeten getroffen worden, zoals antivirus, encryptie, ... zijn maatregelen die niet enkel dienen voor de GDPR, maar een organisatie op zijn geheel beschermen. 
Het is een uitgebreid topic dat niet zal worden besproken in dit onderzoek. 


\subsection{Dataretentie}
bron %ec.europa.eu/info/law/law-topic/data-protection/reform/rules-business-and-organisations/principles-gdpr/how-long-can-data-be-kept-and-it-necessary-update-it_nl
Met dataretentie wordt bedoeld het bijhouden van internetgegevens. Zowel het bijhouden van gegevens in de loop van de tijd, als op welke plaatsen binnen een organisatie gevgevens worden bijgehouden kunnen issues vormen. 

In artikel 5 van de GDPR onder lid e wordt het volgende gesteld over het bijhouden van persoonlijke gegevens in de tijd:

Persoonsgegevens moeten worden bewaard in een vorm die het mogelijk maakt de betrokkenen niet langer te identificeren dan voor de doeleinden waarvoor de persoonsgegevens worden verwerkt noodzakelijk is; \textit{persoonsgegevens mogen voor langere perioden worden opgeslagen voor zover de persoonsgegevens louter met het oog op archivering in het algemeen belang, wetenschappelijk of historisch onderzoek of statistische doeleinden worden verwerkt overeenkomstig artikel 89, lid 1, mits de bij deze verordening vereiste passende technische en organisatorische maatregelen worden getroffen om de rechten en vrijheden van de betrokkene te beschermen (" opslagbeperking");}

Maatregelen voor het eerste deel van artikel 5 kunnen worden genomen door middel van anonimisatie en pseudonimisatie, zoals beschreven in sectie 2.2.5 van dit onderzoek. 

Maar binnen een organisatie moet ook worden nagedacht in welke mate je persoonlijke gegevens zal bewaren over langere tijd. In het algemeen verordent de wetgeving dat \textbf{gegevens voor de kortst mogelijke tijd moeten worden opgeslagen}, met enkele uitzonderingen. Dit is een grijze zone, elke organisatie kan 'kortst mogelijke tijd' anders interpreteren voor hun belangen. Daarom verschaft de Europese Commissie hieromtrent extra informatie. 

%ACTUALISERING

bron %https://ec.europa.eu/info/law/law-topic/data-protection/reform/rules-business-and-organisations/principles-gdpr/how-long-can-data-be-kept-and-it-necessary-update-it_nl
Hoe beslis je als organisatie wat de kortst mogelijke periode is? Hier moet worden rekening gehouden met de redenen waarom de gegevens verwerkt en daarna bijgehouden worden. Soms zijn er wettelijke verplichtingen om voor een lange tijd alles bij te houden(bijvoorbeeld nationale arbeids-, belasting- of fraudebestrijdingswetten die u verplichten om persoonsgegevens over uw werknemers gedurende een bepaalde periode te bewaren, productgarantieduur enz.). 

Als er geen wettelijke verplichtingen zijn moet je dus al organisatie een goede reden hebben waarom je langere tijd gegevens wilt bewaren. \textbf{Het moet een duidelijk belang hebben én het is belangrijk dat je die data actueel houd}t. Anders valt volgens de wetgeving het belang van die data weg, en mag deze niet worden bijgehouden. 
\\ Een voorbeeld wordt gegeven: 
\\ \textit{Gegevens te lang bewaard zonder actualisering.}

\setlength{\leftskip}{1cm}
'Uw onderneming/organisatie exploiteert een wervingsbureau en verzamelt daarvoor cv's van werkzoekenden die, in ruil voor de aangeboden bemiddelingsdiensten, een vergoeding betalen. Uw onderneming/organisatie wenst de gegevens twintig jaar te bewaren maar uw onderneming/organisatie heeft geen maatregelen getroffen om de cv’s te actualiseren. De opslagperiode lijkt niet evenredig met het doel om op korte tot middellange termijn werk voor iemand te vinden. Doordat uw onderneming/bedrijf niet regelmatig om een actualisering van de cv’s vraagt, wordt een deel van de zoekopdrachten na een bepaalde tijd bovendien nutteloos voor de werkzoekende (bijvoorbeeld omdat de betrokkene nieuwe kwalificaties heeft verworven).'

\setlength{\leftskip}{0pt}

Naast de correcte beweegredenen voor het bijhouden van de data, dient dit ook op de correcte manier uitgevoerd te worden. 

% Actualiseren en logging

%De Europese commissie voo
%niet lang houden van data, logging wanner is gedaan, data waar pii in zit automatisch verwijderen! 




\subsection{Pseudo- en anonimisatie van gegevens}
https://gdpr-info.eu/recitals/


De GDPR is van kracht op persoonsgegevens. Een manier om je data te kunnen gebruiken zonder aan de strenge regelgeving te moeten voldoen is door middel van je data te anonimiseren. Dit wil zeggen dat je verkregen data niet langer traceerbaar is tot de specifieke persoon van wie het afkomstig is. Volgens recital 26\footnote{Zie recital 26 in de bijlage.} valt geanonimiseerde data niet meer onder de GDPR, wat uitbreidere mogelijkheden biedt voor verwerking. 

Voorbeeld anonimisatie: bankkaart nummer "4000-1234-4567-9123" wordt weergegeven als "4000-XXXX-XXXX-XX23". 

Voor veel bedrijven is volledig anonimiseren van data minder interessant. Een iets minder drastische oplossing is pseudonimisatie. Hierbij zorg je ervoor dat de uitgevoerde anonimisatie omkeerbaar blijft.  
Bij deze techniek ga je bepaalde velden, die heel duidelijk herleidbaar zijn tot een individu, zoals naam, adres, ... vervangen door een pseudoniem.
Hoewel er bij anonimisatie duidelijk wordt gezegd dat de data niet meer onder de gdpr valt, is dit bij psuedonimisatie niet het geval. De regelgeving stelt dat het een maatregel is die het risico van ongewenst verspreiden van persoonlijke data aanzienelijk kan verkleinen, en daarnaast kan bijdragen tot data protection by design. Maar er wordt ook expliciet vermeld in Recital 28 dat pseudonimisatie de andere maatregelen niet uitsluit, en de data dus nog steeds onder de gdpr valt. 
Er wordt heel duidelijk gesteld in de regelgeving dat psuedonimisatie wordt aangeraden, het een deel kan zijn van data protection by design en by default, indien je de data zo snel mogelijk pseudonimiseert. 

Technische mogelijkheden hoe je op een efficiënte manier data kan pseudonimiseren vindt je in sectie 3: Methodologie. 

https://gdpr-info.eu/recitals/ bijlage
Recital 26
1The principles of data protection should apply to any information concerning an identified or identifiable natural person. 2Personal data which have undergone pseudonymisation, which could be attributed to a natural person by the use of additional information should be considered to be information on an identifiable natural person. 3To determine whether a natural person is identifiable, account should be taken of all the means reasonably likely to be used, such as singling out, either by the controller or by another person to identify the natural person directly or indirectly. 4To ascertain whether means are reasonably likely to be used to identify the natural person, account should be taken of all objective factors, such as the costs of and the amount of time required for identification, taking into consideration the available technology at the time of the processing and technological developments. 5The principles of data protection should therefore not apply to anonymous information, namely information which does not relate to an identified or identifiable natural person or to personal data rendered anonymous in such a manner that the data subject is not or no longer identifiable. 6This Regulation does not therefore concern the processing of such anonymous information, including for statistical or research purposes.

\begin{figure}[h]
	\includegraphics[width=\linewidth]{anonymise.jpg}
	\caption{Data anonymisation schematische voorstelling}
	http://www.anonymizedata.com/
\end{figure}


\subsection{Data access control.}
TRAININGEN 

Een overweging die moet gemaakt worden is welke documenten voor welke personen toegankelijk zijn. Als voorbeeld een kmo die voor bepaalde projecten gegevens verzamelt. Dan moeten enkel de personeelsleden die aan dit project meewerken toegang hebben tot die gegevens.  Vaak zien we echter zo dat er een gemeenschappelijke locatie is waar alle gegevens worden bijgehouden, en die toegankelijk is voor alle werknemers. 
En kan dat personeel de toegankelijke data lokaal kopiëren? Zijn hieromtrent regels opgesteld? Als onderneming moet je hier rekening mee houden. Antwoorden op deze vragen zijn geformuleerd in hoofdstuk 3. 

\subsubsection{Verzoek tot verwijderen}
Het technische deel van dit onderzoek zal hier de focus op leggen. Met name wanneer een gebruiker het verzoek indient om al de gegevens die je als organisatie van die persoon bijhoudt, te verwijderen. De eerste stappen die je dan logischerwijs onderneemt is het anonimiseren van zijn opgeslagen gegevens. Denk aan adres, telefoonnummer, enzoverder. 

 

\section{Scope: verwerken niet-structurele data.}

Gegevens verwijderen uit niet-structurele data is een gevorderde maatregel die technisch uitdagender is dan vorig vermelde maatregelen, en is daarom opgedeeld in een apart hoofdstuk. Hier wordt 
beschreven hoe persoonlijke informatie precies kan voorkomen in niet-structurele data, en welke oplossingen voorhanden zijn om deze te zoeken en eventueel verwijderen. Binnen dit onderzoek
worden die oplossingen ook praktisch getest in hoofdstuk 4. 

\subsection{Persoonlijke informatie in niet-structurele data}
Als een organisatie beslist om informatie over een persoon te anonimiseren of verwijderen, zijn de eerste stappen duidelijk. Alle structurele data verwijderen/anonimiseren. Maar ook in niet-structurele data kan persoonlijke informatie voorkomen. Een voorbeeld: 

Menig organisaties die onderzoek uitvoeren, maken gebruik van openbare fora. Denk aan reacties op een post op Facebook. Wanneer zo'n forum eigendom is van de organisatie, en zij beslissen welke inhoud er online blijft en opgeslagen wordt, moet ook hier nagedacht worden over de GDPR. Stel, een gebruiker reageert op een post over Hiv-remmers het volgende: \textit{”Ik vind merk X niet goed, want ik woon in straat Y in dorp Z en ze komen hier niet leveren aan de deur, waardoor A(mijn zoon 12) er altijd om moet in het weekend.”}
Op basis van dit bericht is het niet moeilijk om te concluderen over wie het gaat, en kan een buitenstaander dus afleiden dat die persoon Hiv-positief is. \\ Wat leidt tot de vraag: Moet dit soort data ook verwijderd worden bij een een verzoek tot verwijderen? Het antwoord hierop is Ja.

 Maar hoe kan dit uitgevoerd worden? Een optie is om een manueel alles te gaan nalezen en zelf te beslissen wat verwijderd moet worden. Dit proces is echter tijdrovend en onefficënt. Daarom werd in dit onderzoek verder op zoek gegaan naar oplossingen, en die zijn gevonden in de vorm van Machine learning en Artificiële intelligentie.


In het recente verleden heeft Artificiële Intelligentie en bij uitbreiding Machine Learning een geweldige evolutie gemaakt. AI, waarbij vele mensen nog veel te vaak denken aan robotjes die als mensen functioneren, staat ongetwijfeld in zijn kinderschoenen. Er zijn al veel toepassingen, ontwikkeld door de 'grotere spelers' uit de informaticawereld. Deze toepassingen leveren bruikbare en betrouwbare resultaten, en zijn voor iedere software developer beschikbaar. Ze kunnen oplossingen bieden voor het probleem dat PII zich in vele verschillende vormen kan aanbieden (structureel en niet-structureel), en niet altijd door scripts wordt herkend.  

\subsection{Aanbieders API's voor PII-herkenning}
Een KMO die wil gebruik maken van AI hoeft dus geen neurale netwerken op te stellen en moeilijke wiskunde berekingen uit te voeren. Een beter optie is om gebruik te maken van bestaande API's die achter de schermen deze artificiële intelligentie gaan uitvoeren. Hiervoor zijn de laatste jaren talloze opties ontwikkeld. In dit onderzoek zullen de twee grootste aanbieders besproken worden. Beide Microsoft en Google hebben een online computing-platform waar internet diensten op worden aangeboden.\textit{Microsoft Azure platform} met \textit{Cognitive Services} en  \textit{Google} met \textit{Google Cloud platform}. In de volgende secties worden deze twee aanbieders met elkaar vergeleken op vlak van functionaliteit.

Dit onderzoek is uitgevoerd in samenwerking met InSites Consulting, die op moment van schrijven reeds gebruik maakt van de services van Microsoft Azure. Vandaar is er een voorkeur voor het werken met Cognitive Services. Verder in dit onderzoek zal echter blijken dat Google Cloud op bepaalde vlakken meer mogelijkheden biedt. Afhankelijk daarvan zal in de methodologie met één van beide, of beide technologieën gewerkt worden. 

\begin{figure}[h]
	\centering
	\begin{subfigure}{0.5\textwidth}
		\centering
		\includegraphics[width=.4\linewidth]{cognitive_services.png}
		\caption{Microsofts' Azure cognitive services}
		\label{fig:sub11}
	\end{subfigure}%
	\begin{subfigure}{0.5\textwidth}
		\centering
		\includegraphics[width=.4\linewidth]{google_cloud_ml.png}
		\caption{Google cloud machine learning.}
		\label{fig:sub22}
	\end{subfigure}
	\caption{De twee grootste aanbieders van een API voor machine learning solutions.}
	\label{fig:test2}
\end{figure}

\subsubsection{Microsoft Azure}
Azure is een online computing platform. In zijn totaliteit biedt het Infrastucture as a Service (=Iaas), Platform as a Service(Paas), en Software as a Service (Saas). Organisaties kunnen het gebruiken voor onder andere (cloud-)opslag, montoring en analyse, netwerken, cloud servers, enzoverder. 
Azure is zeer populair binnen enterprises: volgens cijfers van Microsoft zelf  https://azure.microsoft.com/nl-nl/overview/azure-vs-aws/ maakt meer dan 95 procent van de Fortune 500 bedrijven (= 500 grootste bedrijven in de Verenigde Staten op basis van jaaromzet) gebruik van Azure. Microsoft brengt geen exacte cijfers over het aantal gebruikers, maar uit hun kwartaalcijfers van Q1 2019 
%BRON https://www.microsoft.com/en-us/annualreports/ar2018/annualreport
blijkt dat Azure, met een omzet van 8.6 miljard, goed is voor net geen 30 procent van de totale omzet. 
Dit maakt het voor veel ondernemingen gemakkelijker om aan de slag te gaan met cognitive services, aangezien het een onderdeel is van Azure, die al wereldwijd verspreid is, en dus vermoedelijk niet snel zal verdwijnen. Azure werkt onder het \textit{you pay what you use} principe. Dus voor de meeste bedrijven betekent dit een extra bedrag op hun Azure-rekening, al lijken de prijzen eerder aan de lage kant te zijn, als aanvragen tot verwijdering van data niet frequent voorkomen. \\MARK TO SELF PRICING !
De overige mogelijkheden zijn eindeloos, maar minder van belang in dit onderzoek. 

\subsubsection{Google cloud}
LEZEN/ %https://www.suse.com/media/report/rightscale_2018_state_of_the_cloud_report.pdf

Google cloud services (2008) bestaat iets langer dan MS Azure (2010) BRONNEN, maar is desondanks iets minder populair (vergelijking in aantal users). Dit komt voornamelijk door Microsoft's strategie, die hun cloud meteen richtte op ondernemingen. Op heden hebben beide zeer gelijklopende tools en functionaliteit. Waar Azure iets sterker staat op vlak van ondernemingsgerichte tools, is google koploper bij alles rond analyses, en (vooral van belang voor dit onderzoek): AI en Machine Learning. 

Google beseft ook al een tijdje dat machine learning oplossingen kan bieden voor het beschermen van persoonlijke data. Daarvoor hebben ze een aparte API ontwikkeld: Google Cloud Data Loss Prevention (DLP). Hiermee willen ze tools aanbieden om "gevoelige gegevens beter te begrijpen en beheren". 
https://cloud.google.com/dlp/

\subsection{Foto's: Gezichtsherkenning}
De toepassing van machine-learning die ongetwijfeld het wijdst verspreid en verst geëvolueerd is. Dankzij zijn vele implementaties in onder andere smartphones en beveiligingsapparatuur is gezichtsherkenning een zeer betrouwbare én snelle technologie geworden. Het is dan ook weinig verrassend dat beide aanbieders (Microsoft en Google) het ter beschikking stellen. 

\subsubsection{Cognitive services}
Cognitive services biedt op zijn website een demo aan van wat de API precies kan op vlak van gezichtsherkenning.   


\begin{figure}[h]
	\includegraphics[width=\linewidth]{Face_recogn_micros_1.png}
	\caption{Cognitive services - Face recognition. Links de foto, rechts een JSON-file.}
	\label{fig:cognitive}
\end{figure}

Op \hyperref[fig:cognitive]{figuur 2.5} zie je de gezichtsherkenning van Microsoft in werking. Dit is een online demo beschikbaar voor iedereen op de website van Cognitive services. Links op de figuur staat een foto met een duidelijk herkenbaar gezicht. Rechts staat het verwerkte resultaat van een interpretratie van wat in de foto te zien is, automatisch gegenereerd. Er wordt een JSON-file uitgeprint met alle herkende elementen. Het eerstgevonden element is het face-id. Zodra er een face-id aanwezig in de JSON, betekent dit dat een gezicht herkend is. Daarnaast krijg je nog een heel wat extra elementen, die tot nog toe niet van belang zijn. 

Er is dus heel wat info over de foto, en in het kader van persoonlijke informatie valt hier heel veel in terug te vinden. 

\subsubsection{Google cloud}
Met Google Cloud Vision levert Google een zeer gelijkaardige service. Opnieuw is een online demo beschikbaar die een preview geeft van hoe gezichten op foto's herkend worden. We gebruiken dezelfde foto als in \hyperref[fig:cognitive2]{figuur 2.5}. Het resultaat zie je in figuur 2.7. De interpratie is nagenoeg identiek. Het gezicht werd vlot herkend en er is heel wat extra informatie te zien, zoals 'Joy'. 

\begin{figure}[h]
	\includegraphics[width=\linewidth]{FAce_recogn_google2.png}
	\caption{Cognitive services - Face recognition. Links de foto, rechts een JSON-file met herkende elementen.}
	\label{fig:facerecogn}
\end{figure}


\subsection{Foto's: classificatie en textextractie}
Naast tekst beschikken beide google cloud DLP en MS cognitive services ook over herkenning van elementen in foto's, die kunnen geclassificeerd worden. De API zou bijvoorbeeld gemakkelijk een huis herkennen. Dit biedt dus de mogelijkheid om ook foto's te gaan filteren op persoonlijke informatie naast gezichten. 


\begin{figure}[h]
	\includegraphics[width=\linewidth]{Image_classification_MS_1.png}
	\caption{Cognitive service - Image classification. Links de foto, rechts een JSON-file waarin een huis wordt herkend, zekerheid 0.98. }
	\label{fig:Houserecogn}
\end{figure}

Daarnaast kunnen foto's ook zelf tekst bevatten met persoonlijke info. 
\\ Als voorbeeld: een doktersvoorschrift. Dit is heel persoonlijk, en zou dus moeten kunnen verwijderd worden.
\\ De uitdaging is hier om eerst en vooral de tekst te herkennen van op de foto's, daarna binnen die tekst het te kunnen classificeren als persoonlijke info, en die dan te verbergen.
Er is alvast een tool aanwezig die tekst herkent in foto's. Op die manier kan je de verkregen tekst gaan controleren op parameters als adres, email, enzoverder. (Zie 2.3.4)

\subsection{Persoonlijke informatie in tekst}


\subsection{Gesproken tekst}
Lorem ipsum dolor sit amet, consectetur adipiscing elit, sed do eiusmod tempor incididunt ut labore et dolore magna aliqua. Ut enim ad minim veniam, quis nostrud exercitation ullamco laboris nisi ut aliquip ex ea commodo consequat. Duis aute irure dolor in reprehenderit in voluptate velit esse cillum dolore eu fugiat nulla pariatur. Excepteur sint occaecat cupidatat non proident, sunt in culpa qui officia deserunt mollit anim id est laborum.

