%%=============================================================================
%% Voorwoord
%%=============================================================================

\chapter*{\IfLanguageName{dutch}{Woord vooraf}{Preface}}
\label{ch:voorwoord}

Deze bachelorproef is geschreven als afsluitend element van de opleiding Toegegepaste Informatica aan HoGent. 

Als onderwerp heb ik gekozen om een onderzoek uit te voeren rond de GDPR. Deze keuze is geïnspireerd door actuele gebeurtenissen
die in het nieuws verschenen met betrekking tot de GDPR, gelijktijdig met het begin van dit onderzoek. Deze actuele gebeurtenissen hebben me aan het denken gezet, en ik vroeg me af hoe moeilijk het is voor een organisatie om iedereen zijn gegevens, op een correcte manier, te beschermen. Privacy is een waarde in onze samenleving die, naar mijn mening, hoog in het vaandel moet gedragen worden. 

Specifiek wou ik vanuit mijn expertise (als student Toegepaste Informatica) gaan onderzoeken welke maatregelen exact kunnen genomen worden, en welke hindernissen daarbij moeten overwonnen worden. 
Hiermee hoop ik een onderzoek af te leveren dat nuttig kan zijn voor vele ondernemingen. Daarnaast wil ik de lezer aanzetten tot het nadenken over privacy, en activeren tot het ondernemen van maatregelen. 
 
Dit was de eerste keer dat ik een uitgebreid onderzoek en bijhorende scriptie heb opgesteld, dus kon ik dit uiteraard niet alleen. Het lijkt me dan ook op zijn plaats om enkele mensen te bedanken voor de hulp:

De heer Lieven Smits voor het nalezen van mijn scriptie en het geven van feedback, en Mevrouw Chantal Teerlinck, voor het ondersteunen van de bachelorproef doorheen het hele proces. 

Alle leden van mijn gezin, voor het kritisch nalezen, en een persoonlijke inbreng te geven wat zij anders zouden doen. 

Het volledige team van InSites Consulting, waar ik altijd terecht kon voor vragen, en die me van begin tot eind goede ondersteuning hebben geboden. 

En tot slot een speciaal dankwoord aan de heer Gunter Van de Velde, de verantwoordelijk voor de GDPR binnen InSites Consulting, en de persoon die heeft opgetreden als mijn co-promotor. Gunter stond altijd open voor overleg, wou altijd helpen brainstormen. Hij zorgde voor kritisch commentaar en nieuwe ideeën. Zonder zijn hulp was deze scriptie niet geweest zoals hij nu is. 


%% TODO:
%% Het voorwoord is het enige deel van de bachelorproef waar je vanuit je
%% eigen standpunt (``ik-vorm'') mag schrijven. Je kan hier bv. motiveren
%% waarom jij het onderwerp wil bespreken.
%% Vergeet ook niet te bedanken wie je geholpen/gesteund/... heeft

