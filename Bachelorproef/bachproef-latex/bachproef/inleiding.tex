%%=============================================================================
%% Inleiding
%%=============================================================================

\chapter{\IfLanguageName{dutch}{Inleiding}{Introduction}}
\label{ch:inleiding}

In dit hoofdstuk wordt het werk 'General Data Protection Regulation voor Software as a service met persoonlijke gegevensverwerking: een praktische analyse' ingeleid. Er wordt besproken wat wordt onderzocht en waarom, en wat de doelstellingen zijn van dit onderzoek.  
\section{\IfLanguageName{dutch}{Probleemstelling}{Problem Statement}}
\label{sec:probleemstelling}
Sinds mei 2018 zijn ondernemingen verplicht om te voldoen aan een nieuwe Europese regelgeving: de 'General Data Protection Regulation' of kortweg GDPR. 
Ondernemingen die onder de wettelijk voorwaarden vallen (zie verder) moeten op zoek naar mogelijke manieren om persoonsgegevens die ze bijhouden en verwerken te beveiligen zoals omschreven in de GDPR. Hiervoor zijn talloze stappen te ondernemen, deze stappen zullen in dit onderzoek geanalyseerd worden. 

Algemeen wordt in dit werk gekeken naar KMO's die op geautomatiseerde wijze, op grote\footnote{Wat precies bedoeld wordt met 'groot' leest u verder in dit onderzoek.} schaal gegevens verwerken. 
Voor deze doelgroep kan dit werk een vertrekpunt bieden om een overzicht te krijgen van de mogelijke maatregelen die ze kunnen uitvoeren. 
Specifiek worden deze maatregelen toegepast op InSites Consulting. Er wordt gekeken welke maatregelen reeds uitgevoerd zijn en wat er nog beter kan. 
Als extra-uitgelicht element wordt bekeken hoe InSites gegevens uit niet-gestructureerde data kan beschermen. 

\section{\IfLanguageName{dutch}{Onderzoeksvraag}{Research question}}
\label{sec:onderzoeksvraag}

Welke maatregelen kunnen op organisatorisch en vooral op technisch (softwarematig) vlak genomen worden om een ondermening te laten voldoen aan de wettelijke normen binnen de GDPR? Aan welke van deze maatregelen voldoet InSites Consulting reeds als onderneming, en wat kan nog beter? Welke oplossingen zijn voorhanden voor de maatregelen die nog niet genomen zijn?

\section{\IfLanguageName{dutch}{Onderzoeksdoelstelling}{Research objective}}
\label{sec:onderzoeksdoelstelling}

Het doel is om een zo breed mogelijke waaier van maatregelen aan te bieden om met gegevens om te gaan zoals in de GDPR omschreven. 
Een bijkomend doel is om specifiek maatregelen te zoeken voor problemen die voorkomen bij InSites Consulting, en die maatregelen verder uit te werken. 

\section{\IfLanguageName{dutch}{Opzet van deze bachelorproef}{Structure of this bachelor thesis}}
\label{sec:opzet-bachelorproef}

% Het is gebruikelijk aan het einde van de inleiding een overzicht te
% geven van de opbouw van de rest van de tekst. Deze sectie bevat al een aanzet
% die je kan aanvullen/aanpassen in functie van je eigen tekst.

De rest van deze bachelorproef is als volgt opgebouwd:

In Hoofdstuk~\ref{ch:stand-van-zaken} wordt een overzicht gegeven van de stand van zaken binnen het onderzoeksdomein, op basis van een literatuurstudie. Hier wordt vooral het theoretisch aspect van de GDPR toegelicht. 

In Hoofdstuk~\ref{ch:Maatregelen} worden mogelijke maatregelen voorgesteld die een bedrijf kan ondernemen als antwoord op de onderzoeksvragen. 

In Hoofdstuk~\ref{ch:methodologie} wordt de methodologie toegelicht enkele maatregelen uit hoofdstuk 3 verder uitgewerkt.

% TODO: Vul hier aan voor je eigen hoofstukken, één of twee zinnen per hoofdstuk

In Hoofdstuk~\ref{ch:conclusie}, tenslotte, wordt de conclusie gegeven en een antwoord geformuleerd op de onderzoeksvragen. Daarbij wordt ook een aanzet gegeven voor toekomstig onderzoek binnen dit domein.