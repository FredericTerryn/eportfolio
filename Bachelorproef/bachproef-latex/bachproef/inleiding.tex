%%=============================================================================
%% Inleiding
%%=============================================================================

\chapter{\IfLanguageName{dutch}{Inleiding}{Introduction}}
\label{ch:inleiding}

De inleiding moet de lezer net genoeg informatie verschaffen om het onderwerp te begrijpen en in te zien waarom de onderzoeksvraag de moeite waard is om te onderzoeken. In de inleiding ga je literatuurverwijzingen beperken, zodat de tekst vlot leesbaar blijft. Je kan de inleiding verder onderverdelen in secties als dit de tekst verduidelijkt. Zaken die aan bod kunnen komen in de inleiding~\autocite{Pollefliet2011}:

\begin{itemize}
  \item context, achtergrond
  \item afbakenen van het onderwerp
  \item verantwoording van het onderwerp, methodologie
  \item probleemstelling
  \item onderzoeksdoelstelling
  \item onderzoeksvraag
  \item \ldots
\end{itemize}

\section{\IfLanguageName{dutch}{Probleemstelling}{Problem Statement}}
\label{sec:probleemstelling}

Ondernemingen die wettelijk verplicht zijn te voldoen aan de GDPR moeten sinds mei 2018 op zoek naar alle mogelijke manieren persoonsgegevens die ze bijhouden en verwerken te beveiligen. Hiervoor zijn (zoals verder in dit onderzoek zal blijken) talloze stappen te ondernemen. Voor een onderneming is het belangrijk diepgaand te onderzoeken op welke manieren er kan worden omgesprongen met persoonlijke gegevens, en hoe ze die op een goede manier kunnen/mogen bewaren. 

Algemeen wordt in dit werk gekeken naar KMO's die op geautomatiseerde wijze, op grote\footnote{Wat precies bedoeld wordt met 'groot' leest u verder in dit onderzoek.} schaal gegevens verwerken. 
Voor deze doelgroep kan dit werk een vertrekpunt bieden om een overzicht te krijgen van de mogelijke maatregelen die ze kunnen uitvoeren. 
Specifiek worden deze maatregelen toegepast op InSites Consulting. Er wordt gekeken welke maatregelen reeds uitgevoerd zijn, en wat er nog beter kan. 
Als extra-uitgelicht element wordt bekeken hoe InSites gegevens uit niet-gestructureerde data kan beschermen. 

\section{\IfLanguageName{dutch}{Onderzoeksvraag}{Research question}}
\label{sec:onderzoeksvraag}

Welke maatregelen kunnen op organisatorisch en vooral op technisch (softwarematig) vlak genomen worden om een ondermening te laten voldoen aan de wettelijke normen binnen de GDPR? Aan welke van deze maatregelen voldoet InSites Consulting reeds als onderneming, en wat kan nog beter? Welke oplossingen zijn voorhanden voor de maatregelen die nog niet genomen zijn. 

\section{\IfLanguageName{dutch}{Onderzoeksdoelstelling}{Research objective}}
\label{sec:onderzoeksdoelstelling}

Wat is het beoogde resultaat van je bachelorproef? Wat zijn de criteria voor succes? Beschrijf die zo concreet mogelijk. Gaat het bv. om een proof-of-concept, een prototype, een verslag met aanbevelingen, een vergelijkende studie, enz.

\section{\IfLanguageName{dutch}{Opzet van deze bachelorproef}{Structure of this bachelor thesis}}
\label{sec:opzet-bachelorproef}

% Het is gebruikelijk aan het einde van de inleiding een overzicht te
% geven van de opbouw van de rest van de tekst. Deze sectie bevat al een aanzet
% die je kan aanvullen/aanpassen in functie van je eigen tekst.

De rest van deze bachelorproef is als volgt opgebouwd:

In Hoofdstuk~\ref{ch:stand-van-zaken} wordt een overzicht gegeven van de stand van zaken binnen het onderzoeksdomein, op basis van een literatuurstudie. Hier wordt vooral het theoretisch aspect van de GDPR toegelicht. 

In Hoofdstuk~\ref{ch:Maatregelen} worden de mogelijke maatregelen die een bedrijf kan ondernemen om beter te voldoen aan de GDPR omgeschreven. 

In Hoofdstuk~\ref{ch:methodologie} wordt de methodologie toegelicht en worden de gebruikte onderzoekstechnieken besproken om een antwoord te kunnen formuleren op de onderzoeksvragen.

% TODO: Vul hier aan voor je eigen hoofstukken, één of twee zinnen per hoofdstuk

In Hoofdstuk~\ref{ch:conclusie}, tenslotte, wordt de conclusie gegeven en een antwoord geformuleerd op de onderzoeksvragen. Daarbij wordt ook een aanzet gegeven voor toekomstig onderzoek binnen dit domein.